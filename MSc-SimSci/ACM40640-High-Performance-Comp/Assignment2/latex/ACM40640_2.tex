\documentclass[a4paper]{article}

\usepackage[english]{babel}
\usepackage[utf8]{inputenc}
\usepackage[titletoc, toc]{appendix}
\usepackage{amsmath}
\usepackage{graphicx}
\usepackage{mdframed}
\usepackage{cancel}
\usepackage{caption}
\usepackage{lipsum}
\usepackage{listings}
\usepackage{float}
\usepackage{csvsimple}
\usepackage{array}
\usepackage[table]{xcolor}
\usepackage[colorinlistoftodos]{todonotes}

\definecolor{mGreen}{rgb}{0,0.6,0}
\definecolor{mGray}{rgb}{0.5,0.5,0.5}
\definecolor{mPurple}{rgb}{0.58,0,0.82}
\definecolor{backgroundColour}{rgb}{0.95,0.95,0.92}

\title{ACM40640 High Performance Comp \\ Assignment 2}

\author{Ian Towey \\ \\ 04128591}

\date{\today}

\lstdefinestyle{custom_py_style}{
  belowcaptionskip=1\baselineskip,
  breaklines=true,
  frame=L,
  xleftmargin=\parindent,
  language=Python,
  showstringspaces=false,
  basicstyle=\footnotesize\ttfamily,
  keywordstyle=\bfseries\color{green!40!black},
  commentstyle=\itshape\color{purple!40!black},
  identifierstyle=\color{blue},
  stringstyle=\color{orange}
}

\lstdefinestyle{CStyle}{
    backgroundcolor=\color{backgroundColour},   
    commentstyle=\color{mGreen},
    keywordstyle=\color{magenta},
    numberstyle=\tiny\color{mGray},
    stringstyle=\color{mPurple},
    basicstyle=\footnotesize,
    breakatwhitespace=false,         
    breaklines=true,                 
    captionpos=t,                    
    keepspaces=true,                 
    numbers=left,                    
    numbersep=5pt,                  
    showspaces=false,                
    showstringspaces=false,
    showtabs=false,                  
    tabsize=2,
    language=C
}


% Number the subsubsections and include them in the TOC
\setcounter{secnumdepth}{3}
\setcounter{tocdepth}{3}
\setcounter{section}{-1}

\newenvironment{aside}
  {\begin{mdframed}[style=0,%
      leftline=false,rightline=false,leftmargin=2em,rightmargin=2em,%
          innerleftmargin=0pt,innerrightmargin=0pt,linewidth=0.5pt,%
      skipabove=7pt,skipbelow=7pt]\footnotesize}
  {\end{mdframed}}

\begin{document}

  \maketitle

  \begin{abstract}
    MPI code
  \end{abstract}

\tableofcontents

\newpage
\section{Communication in a ring}

File main\_ring.c (see Appendix Listing 1 or accompanying file) demonstrates communication in a ring using MPI.
Each processor sends a message to the its neighbour with rank one greater than its rank module the number of processors.
The message sent along is the accumlated rank of the number of processors in the ring

e.g for a ring of size 5 starting at processor with rank 0
\begin{center}
 $0  \xrightarrow[\text{     0     }]{\text{     msg = 0     }} 1 \xrightarrow[\text{     0+1     }]{\text{     msg = 1     }} 2 \xrightarrow[\text{     0+1+2     }]{\text{     msg = 3     }} 3 \xrightarrow[\text{     0+1+2+3     }]{\text{     msg = 6     }} 4 \xrightarrow[\text{     0+1+2+3+4     }]{\text{     msg = 10     }} 0$
\end{center}


e.g for a ring of size 6 starting at processor with rank 2
\begin{center}
 $2  \xrightarrow[\text{     2     }]{\text{     msg = 2     }} 3 \xrightarrow[\text{     2+3     }]{\text{     msg = 5     }} 4 \xrightarrow[\text{     2+3+4     }]{\text{     msg = 9     }} 5 \xrightarrow[\text{     2+3+4+5     }]{\text{     msg = 14     }} 0 \xrightarrow[\text{     2+3+4+5+0     }]{\text{     msg = 14     }} 1 \xrightarrow[\text{     2+3+4+5+0+1     }]{\text{     msg = 15     }} 2$
\end{center}

The origin processor is a commandline parameter, sample run of the program for a ring of size 6 starting at processor 2

\begin{verbatim}
$ cd /ichec/home/users/ph5xx10/PH504/assignment2/code/q1
$ module load dev intel/2015-u3
$ qsub -I -l walltime=00:20:00 -l nodes=3:ppn=24 -N mma -j oe -r n -A ph5xx
$ mpicc -g -o main main_ring.c -lm
$ mpirun -n 6 main 2

-->>2 sending '2' to 3

3 received: '2' from 2			3 sending '5' to 4

4 received: '5' from 3			4 sending '9' to 5

5 received: '9' from 4			5 sending '14' to 0

0 received: '14' from 5			0 sending '14' to 1

1 received: '14' from 0			1 sending '15' to 2

<<--2 received '15' from 1


Finished :: sum of ranks from 0 to 5 = 15

\end{verbatim}

The program uses asynchronous communication between the processors to active this using MPI function MPI\_Isend, MPI\_Irecv, MPI\_Wait
\lstinputlisting[language=Python,style=CStyle, caption = Asynchronous Communication Calls,firstline=42, lastline=48]{/home/ian/Desktop/ACM40640-High-Performance-Comp/Assignment2/assignment2/main_ring.c}


\newpage
\section{Finding Determinent using Cramers Rule}
File main\_det.c calculates the determinent of a 5x5 matrix using cramers rule. 5 Threads are used to calculate the determinant where each thread calculates the determinant 
of a 4x4 matrix of the matrix A. The method then uses  a recursive definition for the determinant of an n x n matrix, the minor expansion formula.

\begin{verbatim}
$ cd /ichec/home/users/ph5xx10/PH504/assignment2/code/q2
$ module load dev intel/2015-u3
$ qsub -I -l walltime=00:20:00 -l nodes=3:ppn=24 -N mma -j oe -r n -A ph5xx
$ mpicc -g -o main main_det.c -lm
$ mpirun -n 5 main

process 4, partial determinant -0.00386718112244897883
process 0, partial determinant -0.01440665145577631201
process 1, partial determinant -0.00361170218748875158
process 2, partial determinant -0.00415480520912068444
process 3, partial determinant -0.00409046542999916014
****************************************************** 
****** Matrix  (n=5)                ********** 
****************************************************** 

	1.000	-0.500	-0.333	-0.250	-0.200	
	-0.500	0.333	-0.250	-0.200	-0.167	
	-0.333	-0.250	0.200	-0.167	-0.143	
	-0.250	-0.200	-0.167	0.143	-0.125	
	-0.200	-0.167	-0.143	-0.125	0.111	

****************************************************** 
full determinant -0.03013080540483388525
******************************************************

\end{verbatim}

The processor figures out which 4x4 submatrix of the 5x5 matrix using its rank. The processor with rank 0, aggregates the minors of the determinent from the 5 processors using MPI\_Reduce

\lstinputlisting[language=Python,style=CStyle, caption = MPI\_Reduce ,firstline=40, lastline=46]{/home/ian/Desktop/ACM40640-High-Performance-Comp/Assignment2/assignment2/code/q2/main_det.c}


\newpage
\section{Finding Deadlock}

Not attempted



\end{document}
