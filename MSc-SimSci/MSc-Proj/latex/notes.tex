\documentclass[a4paper]{article}

\usepackage[english]{babel}
\usepackage[utf8]{inputenc}
\usepackage[titletoc, toc]{appendix}
\usepackage{amsmath}
\usepackage{amsfonts}
\usepackage{graphicx}
\usepackage{mdframed}
\usepackage{cancel}
\usepackage{caption}
\usepackage{lipsum}
\usepackage{listings}
\usepackage{float}
\usepackage[colorinlistoftodos]{todonotes}

\title{MSc Project Notes}

\author{Ian Towey \\ \\ 04128591}

\date{\today}

\lstdefinestyle{custom_py_style}{
  belowcaptionskip=1\baselineskip,
  breaklines=true,
  frame=L,
  xleftmargin=\parindent,
  language=Python,
  showstringspaces=false,
  basicstyle=\footnotesize\ttfamily,
  keywordstyle=\bfseries\color{green!40!black},
  commentstyle=\itshape\color{purple!40!black},
  identifierstyle=\color{blue},
  stringstyle=\color{orange}
}

% Number the subsubsections and include them in the TOC
\setcounter{secnumdepth}{3}
\setcounter{tocdepth}{3}
% \setcounter{section}{-1}

% Partial derivative
\newcommand*{\pd}[3][]{\ensuremath{\frac{\partial^{#1} #2}{\partial #3}}}

\newenvironment{aside}
  {\begin{mdframed}[style=0,%
      leftline=false,rightline=false,leftmargin=2em,rightmargin=2em,%
          innerleftmargin=0pt,innerrightmargin=0pt,linewidth=0.5pt,%
      skipabove=7pt,skipbelow=7pt]\footnotesize}
  {\end{mdframed}}

\begin{document}

  \maketitle

\tableofcontents

\newpage
\section{Background}

In fluid mechanics, mixing refers to the dispersion of a tracer by the twin actions of advection and diffusion. The relevant mathematical model is the advection-diffusion equation, for a given prescribed velocity field. In two limiting cases (no advection or no diffusion), analytical methods can be used to characterize solutions of the model. This project looks at the limiting case of no diffusion.
\par
\vspace{5mm}
Specifically, we look at random flows, and try to characterize the mixing based on properties of the associated rate-of-strain tensor.  As a result of this insight, the original partial differential equation model reduces down to a pair of simple stochastic differential equations.  The overall aim of the project is to "fit" solutions of these stochastic differential equations to full-scale numerical simulations of the advection-diffusion equation.
\par
\vspace{5mm}
Specifically, the project has the following three aims:

\begin{itemize}
  \item Understand the derivation of the reduced stochastic-differential equation system; be able to solve such systems numerically.
  \item Generate direct numerical simulations of the full-scale advection-diffusion equation with random flows.
  \item Use Uncertainty Quantification to estimate the parameters of the reduced stochastic-differential equation system, and thereby "fit" solutions of this reduced model to the data generated by (2).
\end{itemize}

\newpage
\section{Mathematical Model}

The convection–diffusion equation is a combination of the diffusion (PDE heat equation) and convection (advection) equations, and describes physical phenomena where particles, energy, or other physical quantities are transferred inside 
a physical system due to two processes: diffusion and convection. Depending on context, the same equation can be called the advection–diffusion equation, drift–diffusion equation,or (generic) scalar transport equation.

\begin{equation}
\theta_{t} + \mathbf{u} \cdot \nabla \theta = \kappa \triangle \theta
\end{equation}

\begin{itemize}
 \item $\theta$  is the variable of interest (species concentration for mass transfer, temperature for heat transfer)
 \item $\kappa$ is the diffusivity (also called diffusion coefficient), such as mass diffusivity for particle motion or thermal diffusivity for heat transport,
 \item $\mathbf{u}$  is the velocity field that the quantity is moving with. It is function of time and space. For example, in advection, $\theta$ might be the concentration of salt in a river, and then $\mathbf{u}$ would be the velocity of the water flow, $\mathbf{u}$ =f(time, location). Another example, $\theta$ might be the concentration of small bubbles in a calm lake, and then $\mathbf{u}$ would be the velocity of bubbles rising towards the surface by buoyancy (see below) depending on time and location of the bubble. For multiphase flows and flows in porous media, $\mathbf{u}$ is the (hypothetical) superficial velocity.
 \item $\nabla$  represents gradient and $\nabla \cdot$ represents divergence. $\triangle$ is the the Laplace operator $\triangle \theta = \nabla^{2} \theta = \nabla \cdot \nabla \theta = $ defined as the divergence of the gradient of $\theta$. In this equation, $\nabla \theta$ represents concentration gradient.
\end{itemize}

Setting the value of $\kappa = 0$ then the vector field $\mathsf{B} = (-\theta_{y}, \theta_{x})$ satisfies 

\begin{equation}
\mathsf{B}_{t} + \mathbf{u} \cdot \nabla \mathsf{B} = \mathsf{B} \cdot \nabla \mathsf{u}
\end{equation}

\newpage
The Navier-Stokes equations are the complete equations of mootion for a viscous Newtonian fluid. For an incompressiable fluid $(\nabla \cdot \mathbf(V) = 0)$
And the NS equation reduces to 

\begin{equation}
\rho \mathsf{U}_{t} = - \nabla p - \rho \nabla \psi + \mu \nabla^{2} U
\end{equation}

taking the curl of this 

\section{Vorticity}

The vorticity is defined as the curl of the velocity field.

\begin{equation}
\omega = \nabla \times u
\end{equation}

Where $\omega$ and $u$ are vectory fields.

Phsically, vorticity is twice the angular momentum. And from the solenoidal condition, the divergence of the vorticity is zero. $\nabla \cdot \omega = 0$, as $\nabla \cdot (\nabla \times u) = 0$ for any vector $u$ as this is an incompressible flow.

\subsection{Dervation of the Vorticity transport equation}

Begin with the incompressible NS equations


%\begin{equation}
%\pd{u}{t}=D\pd[2]{u}{x^2} \hspace{15mm} x \in (0, L)
%\end{equation}
%\begin{equation}
%\pd{u}{t}=D\pd[2]{u}{x^2} \hspace{15mm} x \in (0, L)
%\end{equation}


\newpage
\section{Glossary}

\begin{itemize}
  \item $\mathbf{Advection}$ is the transport of a substance by bulk motion. In meteorology and physical oceanography, advection often refers to the horizontal transport of some property of the atmosphere or ocean, such as heat, humidity or salinity, and convection generally refers to vertical transport (vertical advection). https://www.youtube.com/watch?v=GoRF6YyabrQ
  \item $\mathbf{Diffusion}$ is the net movement of molecules or atoms from a region of high concentration (or high chemical potential) to a region of low concentration (or low chemical potential) as a result of random motion of the molecules or atoms. Diffusion is driven by a gradient in chemical potential of the diffusing species.
  \item $\mathbf{Convection}$ is the heat transfer due to bulk movement of molecules within fluids such as gases and liquids, including molten rock (rheid). Convection takes place through advection, diffusion or both. 
  \item In continuum mechanics, the $\mathbf{strain-rate-tensor}$ is a physical quantity that describes the rate of change of the deformation of a material in the neighborhood of a certain point, at a certain moment of time. It can be defined as the derivative of the strain tensor with respect to time, or as the symmetric component of the gradient (derivative with respect to position) of the flow velocity.
  \item In continuum mechanics, the $\mathbf{vorticity}$ is a pseudovector field that describes the local spinning motion of a continuum near some point (the tendency of something to rotate[1]), as would be seen by an observer located at that point and traveling along with the flow.
\end{itemize}


\end{document}
% 