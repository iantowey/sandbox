\chapter{Theory and Methods} \label{ch:theory}

\section{Vorticity} \label{sec:theory-v}
The Navier-Stokes equations are the complete equations of motion for a viscous Newtonian fluid. 
\begin{equation}
\label{eqn:ns}
\frac{\partial \mathbf{u}}{\partial t} + (\mathbf{u} \cdot \nabla)\cdot \mathbf{u} = - \nabla p + (-1)^p \nu_p \nabla^{2 p} \mathbf{u} + \mathbf{F}(\mathbf{x},t) - \mathbf{D}(\mathbf{x},t) 
\end{equation}
In this setting, a forcing term $\mathbf{F}$ and dissapative term $\mathbf{D}$ are added, these are added to stop energy build up at small scales in the direct numercal simulation. 
The viscoity term is taken to be hyperviscous of order $p$. This hyperviscosity also helps to stablise the direct numercal simulation of \ref{eqn:ns} and avoid singular solutions.
The flow is assumed to be incompressible, $\nabla \cdot \mathbf{u} = 0$.

The numerical simulation is not performed directly against the Navier-stokes equation. Instead the simulated data is generated from solving the 2-D Vorticity equation as we are intersted in the orientation dynamics of the fluid elements, and the 
vorticity provides up with the spinning of a fluid element, from which simulated probability density function of the lyapunov exponent can be extracted.

The following outlines the derivation of the vorticity transport eqaution from equation \ref{eqn:ns}. The vorticity equation can be derived by 

\begin{itemize}
 \item takes the curl of equation \ref{eqn:ns}
 \item defining $\mathbf{\omega} = \nabla \times \mathbf{u}$
\end{itemize}
\begin{align}
\nabla \times \bigg[ \frac{\partial \mathbf{u}}{\partial t} + (\mathbf{u} \cdot \nabla)\cdot \mathbf{u} &= - \nabla p + (-1)^p \nu_p \nabla^{2 p} \mathbf{u} + \mathbf{F}(\mathbf{x},t) - \mathbf{D}(\mathbf{x},t) \bigg] \label{eqn:vort1}\\
\frac{\partial \omega}{\partial t} + \underbrace{\nabla \times \bigg[ (\mathbf{u} \cdot \nabla)\cdot \mathbf{u} \bigg]}_\text{Convective term} &= \underbrace{\nabla \times \bigg[- \nabla p\bigg]}_\text{Pressure gradient} + \underbrace{\nabla \times \bigg[(-1)^p \nu_p \nabla^{2 p} \mathbf{u}\bigg]}_\text{Hyperviscosity} + \nabla \times \bigg[\mathbf{F}(\mathbf{x},t)\bigg] - \nabla \times \bigg[\mathbf{D}(\mathbf{x},t) \bigg]  \label{eqn:vort2}
\end{align}
\fbox{
  \parbox{
    \textwidth}{Using the following vector identities and results for any vectors $\omega$ and $\mathbf{u}$
    \begin{enumerate}
    \item incompressible fluid, $\nabla \cdot \mathbf{u} = 0$
    \item for a scalar, the curl of the gradient of a scalar is zero, $\phi$, $\nabla \times \nabla \phi = 0$
    \item the div of the curl of a vector is zero, $\nabla \cdot \omega = \nabla \cdot (\nabla \times \mathbf{u}) = 0$
    \item $\frac{1}{2}\nabla(\mathbf{u} \cdot \mathbf{u}) = (\mathbf{u} \cdot \nabla)\mathbf{u} + \mathbf{u} \times (\nabla \times \mathbf{u})$
    \item $\nabla \times (\mathbf{u} \times \omega) = (\omega \cdot \nabla)\mathbf{u} - (\mathbf{u} \cdot \nabla)\omega + \mathbf{u}\underbrace{\nabla\cdot \omega}_\text{$= 0\\$ (from 3)}  -\omega \underbrace{\nabla \cdot \mathbf{u}}_\text{$= 0\\$ (from 1)}$
    \end{enumerate}
  }
}
\\
\\
Applying these to the Convective term in \ref{eqn:vort2} yields 
\begin{align*}
 \nabla \times \bigg[ (\mathbf{u} \cdot \nabla)\cdot \mathbf{u} \bigg] &= \nabla \times \bigg[ \frac{1}{2}\nabla(\mathbf{u} \cdot \mathbf{u}) - \mathbf{u} \times (\underbrace{\nabla \times \mathbf{u}}_\text{$\omega$}) \bigg]\\
 &= \frac{1}{2} \underbrace{\nabla \times \nabla(\mathbf{u} \cdot \mathbf{u})}_\text{$= 0$ (from 2)} - \nabla \times (\mathbf{u} \times \omega) \\
 &= (\omega \cdot \nabla)\mathbf{u} - (\mathbf{u} \cdot \nabla)\omega
\end{align*}

The pressure term in \ref{eqn:vort2} goes away as the curl of the gradient of a scalar is zero from (2). 

\begin{equation}
\label{eqn:vort}
\frac{\partial \omega}{\partial t} + (\mathbf{u} \cdot \nabla)\omega - (\omega \cdot \nabla)\mathbf{u} = (-1)^p \nu_p \nabla^{2 p} \omega + \nabla \times \bigg[\mathbf{F}(\mathbf{x},t)\bigg] - \nabla \times \bigg[\mathbf{D}(\mathbf{x},t) \bigg]
\end{equation}
For 2-d flows, vortex stretching is absent since $\mathbf{u} = u(x,y)\mathbf{e}_{x} + v(x,y)\mathbf{e}_{y}$ and $\mathbf{\omega} = \omega(x,y)\mathbf{e}_{z}$, therfore the vortex stretching term is zero $(\omega \cdot \nabla)\mathbf{u} = 0$
and letting $\nabla \times \mathbf{F}(\mathbf{x},t) = \mathbf{Q}$ and $\nabla \times \mathbf{D}(\mathbf{x},t) = \mathbf{N}$, \ref{eqn:vort} is resolved to 
\begin{equation}
\label{eqn:vort_mat}
\frac{\partial \omega}{\partial t} + (\mathbf{u} \cdot \nabla)\omega = (-1)^p \nu_p \nabla^{2 p} \omega + \mathbf{Q} - \mathbf{N}
\end{equation}
For the purposes of the direct numerical simulation we will define the $\mathbf{D}$ and $\mathbf{N}$ terms here as they will be required when outlining the numerical algorithm next.
Defining the damping term as a constant, $\mathbf{N} = \nu_{0}\omega$ and from \cite{lilly} the forcing term is defined by the difference equation $\mathbf{Q}_{n+1} = R_{n}\mathbf{Q}_{n} + (1 - R^{2}_{n})^{1/2}x_{*}$ where $R_{n}$ is a dimenionless correlation coefficient
and $x_{*} \sim \mathcal{N}(0,1)$, this forcing term is chosen so that a steady turbulent flow is produced in a long running simulation.

Before solving the vorticity equation, we outline some assumptions that aid and simplify the numerical simulation.
$\omega$ is defined on the 2-d grid domain $[0,L] \times [0,L]$ and $\omega$ is periodic over a wavelength $L$. This yields the initial conditions.
\begin{align*}
\omega(x,y,t) = \omega(x + L,y,t)	\\
\omega(x,y,t) = \omega(x,y + L,t)	
\end{align*}
Discretizing \ref{eqn:vort_mat} and adding in the defined terms for damping and forcing yeilds the n-th time-step difference equation
\begin{equation}
\label{eqn:vort_mat_diff_eqn}
\frac{\omega_{n+1} - \omega_{n}}{\Delta t} + \left(\mathbf{u} \cdot \nabla \omega \right)_{n} = (-1)^p \nu_p \nabla^{2p} \left(\frac{\omega_{n+1} - \omega_{n}}{2} \right) + \bigg[R_{n}Q_{n} + (1 - R^{2}_{n})^{1/2}x_{*}\bigg] - \nu_{0} \omega_{n}
\end{equation}
Since we have assumed periodic boundary conditions , \ref{eqn:vort_mat_diff_eqn} can be transformed to an equation in fourier space and solved using a simpler DFT (which used the FFT for optimal performance)
approach rather than a finite-dufference methods.

In Fourier space, the forcing term is also subject to a further constraint, we want to inject energy into the system only within a certain range to produce a steady flow, see \cite{main} and \cite{lilly}. We defined the binary scaling matrix $M^{*}$, with entries
\begin{align*}
 M^{*}_{ij} = 
 \begin{cases}
    1 & \text{if $k_{min}\frac{2\pi}{L} < kx^{2}_{i} + ky^{2}_{j} < k_{max}\frac{2\pi}{L}$} \\
    0 & \text{otherwise} 
  \end{cases}
\end{align*}
This makes the forcing term non-zero only in the specified range in fourier space.

Another issue with computation in fourier space is that multiplication of two wave numbers can produce a number that is smaller than its factors or a wave numbers with infinite value. A method known as dealiasing is 
used to resolved this, this is discussed in depth in \cite{bowman},.
The dealiasing method used here is known as the $2/3$-rule, this truncates the computed waves numbers which blowup. The  binary matrix $D^{*}$ implements the truncating 2/3-rule. 
\begin{align*}
 D^{*}_{ij} = 
 \begin{cases}
    1 & \text{if $|kx_{i}| < \frac{2}{3}max_{1 \leq i \leq N}kx_{i} \cap |ky_{j}| < \frac{2}{3}max_{1 \leq j \leq N}ky_{j}$} \\
    0 & \text{otherwise} 
 \end{cases}
\end{align*}

Adding these yields the final representation of the numerical algorithm in wavespace. The wavenumber are then transformed back to real numbers using a reverse fourier transform.

\begin{equation}
\label{eqn:vort_mat_diff_eqn_fourier}
\hat{\omega}_{n+1} = D^{*} \Biggr[ \frac{\hat{\omega}_{n} - {\Delta t} \Bigg( u \hat{\cdot} \omega  + \frac{|k^{2p}| \hat{\omega}_{n}}{2} + M^{*}\bigg[R_{n}Q_{n} + (1 - R^{2}_{n})^{1/2}x_{*}\bigg] - \nu_{0} \hat{\omega_{n}} \Bigg)}{1 + \frac{\nu_{p}|k^{2p}\Delta t|}{2}} \Biggr]
\end{equation}



\section{Advection-Diffusion equation \& relation to vorticity} \label{sec:theory}
Now we focus on the velocity of the concentration of the tracer gradient. This is defined using the advection-diffusion equation with hyperdiffusivity to stablise the numerical simulation.

\begin{equation}
\frac{\partial \theta}{\partial t} + \mathbf{u} \cdot \nabla \theta = - (-1)^p \nu_p \nabla^{2p} \theta 
\end{equation}

We need to relate the velocity $\mathbf{u} = (u,v)$ to the vorticity $\omega$. In two-dimensional flow of an incompressible viscous fluid, a stream function may be defined. The stream function $\psi$ is a scalar function defined as

\begin{equation*}
  u = -\frac{\partial \psi}{\partial y}; \hspace*{1cm} v = \frac{\partial \psi}{\partial x}
\end{equation*}

As we are restricting outselves to 2-d. $\omega$ can be reduced to 1-d by computing the vorticity vector

\begin{align*}
  \bar{\omega} &= 
      \begin{bmatrix}
	\hat{x} & \hat{y} & \hat{z} \\
	\partial_{x} & \partial_{y} & \partial_{z} \\
	u(x,y,0,t) & u(x,y,0,t) & 0 
      \end{bmatrix} \\ \\
      &= \hat{x} \cdot 0 - \hat{y} \cdot 0 + \hat{z} \left( \frac{\partial v}{\partial x} - \frac{\partial u}{\partial y} \right)
\end{align*}

Plugging the streamfunctions for $u$ and $v$ above gives;

\begin{align*}
  \omega &= \frac{\partial v}{\partial x} - \frac{\partial u}{\partial y} \\
	 &= \frac{\partial^{2} \psi}{\partial^{2} x} - \frac{\partial^{2} \psi}{\partial^{2} y} \\
	 &= \nabla^{2} \psi
\end{align*}

This relation will allow us to solve the advection diffusion equation in the same numerical simulation as the vorticity.

Using the same assumptions (periodic boundary conditions, scaling issue and dealiasing issue etc) as for the vorticity numerical scheme, a spectral method is also used to solve for $\theta$, the concentration of
the tracer being mixed by the flow.
Again discretizing and transforming to wave space yield the difference equation

\begin{equation}
\hat{\theta}_{n+1} = D^{*} \Biggr[ \frac{\hat{\theta}_{n} -\Delta t \hat{\theta}^{conv}_{n} }{1 + \Delta t(-1)^{p}*((-1)^p)\nu_{p}|\mathbf{k}|^{p}} \Biggr]
\end{equation}

where 
\begin{align*}
\theta^{conv} &= \mathbf{u} \cdot \nabla \theta \\
  &= u \cdot \frac{\partial \theta}{\partial x} - v \cdot \frac{\partial \theta}{\partial y} \\
\end{align*}

Therefore
\begin{align*}
\hat{\theta}^{conv} &= u \cdot \frac{\partial \hat{\theta}}{\partial x} - v \cdot \frac{\partial \hat{\theta}}{\partial y} \\
  &= u \cdot ik_{x}\hat{\theta} - v ik_{y}\hat{\theta} \\
\end{align*}


\section{Orientation Dynamics} \label{sec:theory-od}
The context in which the orientation dynamics model is set, starts with defining the 2-D vector field $\mathcal{B} = (-\theta_{y},\theta_{x})$
Where $\theta$ is the concentration of the passively advected tracer in an incompressible flow and diffusion , as outlined in \cite{main}. 

In this setting, the governing equation is the advection-diffusion
\begin{equation}
  \label{eqn:adv-diff}
  \frac{\partial \theta}{\partial t} + \mathbf{u} \cdot \nabla \theta = 0
\end{equation}

Where $\mathbf{u} = (u,v)$ is the velocity field of the fluid in 2-dimensions.

Investigating the $\theta$ under the action of the advection-diffusion equation \ref{eqn:adv-diff} separately in the $x$ and $y$ direction
\begin{align*}
  \frac{\partial }{\partial t}\frac{\partial \theta}{\partial x} + \mathbf{u} \cdot \nabla \frac{\partial \theta}{\partial x} + \frac{\partial \mathbf{u}}{\partial x} \cdot (\nabla \theta) &= 0 \\
  \frac{\partial }{\partial t}\frac{\partial \theta}{\partial y} + \mathbf{u} \cdot \nabla \frac{\partial \theta}{\partial y} + \frac{\partial \mathbf{u}}{\partial y} \cdot (\nabla \theta) &= 0
\end{align*}

Writing this in matrix notation
\begin{align*}
  \frac{\partial }{\partial t} \left( -\theta_{y} \atop \theta_{x} \right) + \mathbf{u} \cdot \nabla \left( -\theta_{y} \atop \theta_{x} \right) = \left( \mathbf{u}_{y} \cdot (\nabla \theta) \atop -\mathbf{u}_{x} \cdot (\nabla \theta) \right)  
\end{align*}

Subsituting in $\mathcal{B}$ the gradient of the vector field  and multiplying out the RHS
\begin{align*}
  \frac{\partial \mathcal{B}}{\partial t}  + \mathbf{u} \cdot \nabla \mathcal{B} &= 
      \begin{bmatrix}
	 u_{y} \cdot \theta_{x}  & v_{y} \cdot \theta_{y} \\
	- u_{x} \cdot \theta_{x}  & -v_{x} \cdot \theta_{y} 
      \end{bmatrix} \\
  \frac{\partial \mathcal{B}}{\partial t}  + \mathbf{u} \cdot \nabla \mathcal{B} &= 
      \begin{bmatrix}
	-v_{y} & u_{y} \\
	v_{x} & -u_{x}  
      \end{bmatrix}\left( -\theta_{y} \atop \theta_{x} \right) 
\end{align*}
Making use of the incompressiblity condition
\begin{align*}
 \nabla \cdot \mathbf{u} = \frac{\partial u}{\partial x} + \frac{\partial v}{\partial y} = 0 \Rightarrow \frac{\partial u}{\partial x} = -\frac{\partial v}{\partial y}
\end{align*}
and replacing the negative signed values in the matrix on the RHS above yields 
\begin{align*}
  \frac{\partial \mathcal{B}}{\partial t}  + \mathbf{u} \cdot \nabla \mathcal{B} &= 
      \begin{bmatrix}
	u_{x} & u_{y} \\
	v_{x} & v_{y}  
      \end{bmatrix}\left( -\theta_{y} \atop \theta_{x} \right) 
\end{align*}

\begin{equation}
  \label{eqn:B}
   \frac{\partial \mathcal{B}}{\partial t}  + \mathbf{u} \cdot \nabla \mathcal{B} = \mathcal{B} \cdot \nabla \mathbf{u}
\end{equation}
Writing \ref{eqn:B} using the material derivative operator, $\frac{d}{dt} = \frac{\partial}{\partial t} + \mathbf{u} \cdot \nabla $ and computing the dot product of \ref{eqn:B} with $\mathcal{B}$
\begin{align*}
  \mathcal{B}\left( \frac{d \mathcal{B}}{dt} \right) &= \mathcal{B} \left( \mathcal{B} \cdot \nabla \mathbf{u} \right)   \\
  \frac{1}{2}\frac{d}{dt}\mathcal{B}^2  &= \langle \mathcal{B}\,,  \left( \nabla \mathbf{u} \right) \mathcal{B}\rangle   \\
\end{align*}
\fbox{
  \parbox{
    \textwidth}{From matrix theory , any $N \times N$ square matrix $\mathbf{M}$ can be written as a sum of its symmetric and anti-symmetric parts \\
      \hspace*{2cm} $\mathbf{M} = \mathbf{S} + \mathbf{S}^{*}$ \\
      \hspace*{2cm} where \\
      \hspace*{4cm} $\mathbf{S} = \frac{\mathbf{M} + \mathbf{M}^{T}}{2}$ \\
      \hspace*{4cm} $\mathbf{S}^{*} = \frac{\mathbf{M} - \mathbf{M}^{T}}{2}$ 
  }
}
\\
\\
Therefore the 2x2 martix $\nabla \cdot \mathbf{u}$ can be decomposed into the sum of its symmetric and anti-symmetric parts
\begin{align*}
  \frac{1}{2}\frac{d}{dt}\mathcal{B}^2  &= \langle \mathcal{B}\,,  \left[ \left(\nabla \mathbf{u}\right)_{s} + \left(\nabla \mathbf{u}\right)_{s^{*}} \right] \mathcal{B}\rangle   \\
\end{align*}

Since we only care about the symmetric part of the $\nabla \cdot \mathbf{u}$
\begin{align*}
  \frac{1}{2}\frac{d}{dt}|\mathcal{B}|^2  &= \langle \mathcal{B}\,, \left(\nabla \mathbf{u}\right)_{s}  \mathcal{B}\rangle   \\
\end{align*}
\fbox{
  \parbox{\textwidth}
  {
    \begin{align*}
      \left(\nabla \mathbf{u}\right)_{s} &= \frac{1}{2} \left(\nabla \mathbf{u} + \left(\nabla \mathbf{u}\right)^{T} \right)\\
      &= \frac{1}{2} \left(
      \begin{bmatrix}
	u_{x} & u_{y} \\
	v_{x} & v_{y}  
      \end{bmatrix} +
      \begin{bmatrix}
	u_{x} & v_{x} \\
	u_{y} & v_{y}  
      \end{bmatrix} \right) \\
      &= \begin{bmatrix}
	u_{x} & \frac{v_{x}+u_{y}}{2} \\
	\frac{v_{x}+u_{y}}{2} & v_{y} 
      \end{bmatrix} \\
      &= \begin{bmatrix}
	s & d \\
	d & -s
      \end{bmatrix} \\
      &= \mathcal{S} \text{  (Rate-of-strain matrix)}
    \end{align*}		       
  }
}


\begin{equation*}
  \frac{1}{2}\frac{d}{dt}|\mathcal{B}|^2  = \langle \mathcal{B}\,, \mathbf{S}  \mathcal{B}\rangle   
\end{equation*}
Regarding $\mathcal{B}$ as a complex-valued function of the complex variable $z = x + iy$ and $\bar{z} = x - iy$, $\mathcal{B} = \mathcal{B}_{1} + i\mathcal{B}_{2}$
\begin{equation}. 
  \label{eqn:three}
  \frac{1}{2}\frac{d}{dt}|\mathcal{B}|^2  = \left(\mathcal{B}_{1},\mathcal{B}_{2}\right) 
      \begin{bmatrix}
	s & d \\
	d & -s
      \end{bmatrix} \left(\mathcal{B}_{1} \atop \mathcal{B}_{2} \right)
\end{equation}
This produces an equation for the magnitude of the tracer gradient, we want to understand the associated angle $\beta$ of this complex-valued function.

The advection equation for $\beta$ is found by defining $\tan \beta = \frac{\theta_{y}}{\theta_{x}}$, and plugging this into the \ref{eqn:adv-diff}

\begin{align*}
 \partial_{t} \tan \beta + \mathbf{u} \cdot \nabla \tan \beta = 0
\end{align*}
Expanding all terms
\begin{align*}
 \left(\partial_{t} + \mathbf{u} \cdot \nabla \right) \tan \beta &= \frac{1}{\theta^{2}_{x}} \left[ -\theta_{x} \left(\mathcal{B} \cdot \nabla \right)u - \theta_{y} \left(\mathcal{B} \cdot \nabla \right)v \right] \\
 \left(\partial_{t} + \mathbf{u} \cdot \nabla \right) \beta &= \frac{1}{|\mathcal{B}|^{2}} \left[ -\theta_{x} \left(\mathcal{B} \cdot \nabla \right)u - \theta_{y} \left(\mathcal{B} \cdot \nabla \right)v \right] \\
 \left(\partial_{t} + \mathbf{u} \cdot \nabla \right) \beta &= \frac{1}{|\mathcal{B}|^{2}} \left( -\theta_{x}, -\theta_{y} \right) \left[\left( \nabla \mathbf{u}  \right)\mathcal{B} \right] \\
\end{align*}
Expanding $\nabla \mathbf{u}$ into the su of its symmetric and antisymmetric parts \\
\fbox{
  \parbox{\textwidth}
  {
    \begin{align*}
      \nabla \mathbf{u} &= \frac{1}{2} \left(\nabla \mathbf{u} + \left(\nabla \mathbf{u}\right)^{T} \right) - \frac{1}{2} \left(\nabla \mathbf{u} - \left(\nabla \mathbf{u}\right)^{T} \right)\\
      &= \frac{1}{2} \left(
      \begin{bmatrix}
	u_{x} & u_{y} \\
	v_{x} & v_{y}  
      \end{bmatrix} +
      \begin{bmatrix}
	u_{x} & v_{x} \\
	u_{y} & v_{y}  
      \end{bmatrix} \right) 
      +\frac{1}{2} \left(
      \begin{bmatrix}
	u_{x} & u_{y} \\
	v_{x} & v_{y}  
      \end{bmatrix} +
      \begin{bmatrix}
	u_{x} & v_{x} \\
	u_{y} & v_{y}  
      \end{bmatrix} \right) \\
      &= \begin{bmatrix}
	u_{x} & \frac{v_{x}+u_{y}}{2} \\
	\frac{v_{x}+u_{y}}{2} & v_{y} 
      \end{bmatrix} +
      \begin{bmatrix}
	0 & \frac{-v_{x}+u_{y}}{2} \\
	\frac{v_{x}-u_{y}}{2} & 0 
      \end{bmatrix} \\
      &= \begin{bmatrix}
	u_{x} & \frac{v_{x}+u_{y}}{2} \\
	\frac{v_{x}+u_{y}}{2} & v_{y} 
      \end{bmatrix} -
      \begin{bmatrix}
	0 & \frac{v_{x}-u_{y}}{2} \\
	\frac{-(v_{x}-u_{y})}{2} & 0 
      \end{bmatrix} \\
      &= \mathcal{S} - \frac{\omega}{2}      
      \begin{bmatrix}
	0 & 1 \\
	-1 & 0 
      \end{bmatrix} \
    \end{align*}		       
  }
}
\begin{align*}
 \left(\partial_{t} + \mathbf{u} \cdot \nabla \right) \beta &= \frac{1}{|\mathcal{B}|^{2}} \left( -\theta_{x}, -\theta_{y} \right) \left[\left( \mathcal{S} - \frac{\omega}{2}      
      \begin{bmatrix}
	0 & 1 \\
	-1 & 0 
      \end{bmatrix}  \right)\mathcal{B} \right] \\
 \left(\partial_{t} + \mathbf{u} \cdot \nabla \right) \beta &= \frac{1}{|\mathcal{B}|^{2}} \left[ \left( -\theta_{x}, -\theta_{y} \right) \mathcal{S} \mathcal{B}  - \frac{\omega}{2} \left( -\theta_{x}, -\theta_{y} \right)
      \begin{bmatrix}
	0 & 1 \\
	-1 & 0 
      \end{bmatrix}  \mathcal{B} \right] \\
  \left(\partial_{t} + \mathbf{u} \cdot \nabla \right) \beta &= \frac{\omega}{2} - \frac{1}{|\mathcal{B}|^{2}} \left( -\theta_{x}, -\theta_{y} \right) \mathcal{S} \mathcal{B}  \\
\end{align*}
\begin{equation}
  \label{eqn:four}
  \frac{d\beta}{dt} = \frac{\omega}{2} - \frac{1}{|\mathcal{B}|^{2}} \left( \mathcal{B}_{2}, -\mathcal{B}_{1} \right) \begin{bmatrix}
	s & d \\
	d & -s
      \end{bmatrix} \left(\mathcal{B}_{1} \atop \mathcal{B}_{2} \right)  \\
\end{equation}

Finally we want to re-write \ref{eqn:three} in terms of the the rate-of-strain matrix $\mathcal{S}$, which is the symmetric part of matrix $\nabla \mathbf{u}$.

The eigenvalues $\mathcal{S}$ are real 
\begin{align*}
 \lambda_{(+)} &= \sqrt{s^{2} + d^{2}} \\
 \lambda_{(-)} &= - \sqrt{s^{2} + d^{2}} \\
\end{align*}

And orthonormal eigenvectors 
\begin{align*}
 \mathbf{X}_{(+)} &= -\frac{1}{\mathcal{N}} \left( 1 \atop -\alpha + \sqrt{\alpha^{2} + 1} \right)  \\
 \mathbf{X}_{(-)} &= -\frac{1}{\mathcal{N}} \left( \alpha - \sqrt{\alpha^{2} + 1} \atop  1 \right)  \\
\end{align*}
where $\alpha=s/d$ and $\mathcal{N} = \sqrt{2\sqrt{\alpha^2 +1}\left( -\alpha + \sqrt{(\alpha^2 + 1)} \right)}$

Since $\mathbf{X}_{(+)}$ and $\mathbf{X}_{(-)}$ are orthonormal, they can be re-expressed in terms of an angle $\varphi$, the angle $\varphi$ is the angle between the x-axis and the expanding direction of the straining flow.
\begin{align*}
 \mathbf{X}_{(+)} &= \left(\cos \varphi \atop \sin \varphi \right)  \\
 \mathbf{X}_{(-)} &= \left(-\sin \varphi \atop \cos \varphi \right)  \\
\end{align*}

$\varphi$ is the angle we want to model as part of the SDE model introduced later.

Computing inner product $\mathcal{B}$ and $\mathcal{S}\mathcal{B}$ as outlined in \cite{stretch} yields the ODE model for the orientation dynamics
\begin{equation}
 \frac{d}{dt}|\mathbf{B}^{2}| = -2 \lambda \sin \zeta |\mathbf{B}^{2}|
\end{equation}
where the term $\Lambda = -2 \lambda \sin \zeta$ is the growth rate of the gradient (Lypunov exponent), and $\zeta = \beta - \varphi -\frac{1}{4}\pi$

Using the result of \ref{eqn:four}, the rate of change of the angle $\zeta$ is model by the ODE , 
\begin{equation}
  \label{eqn:zeta_ode}
 \frac{d \zeta}{dt} = -2 \lambda \cos \zeta + \omega - 2 \frac{d \varphi}{dt}
\end{equation}

The stochastic differential equation system outlined next aims to model $\Lambda$ and $\zeta$.

\section{Stochastic Differential Equations Model}
In this setting we denote the angle $\zeta$ = X and the $\lambda = \mu$, so rewritting \ref{eqn:zeta_ode} as the ODE
\begin{equation}
  \label{eqn:x_sde}
 \frac{d X}{dt} = -2 \mu \cos X + \omega - 2 \frac{d \varphi}{dt}
\end{equation}

The terms $\omega$ and $\mu$ are modelled as two stochastic differential eqautions, decomposed into mean and random components. 
\begin{align*}
\mu &= \mu_{0} + Y(t) \\
\frac{\omega}{2} -  \frac{d \varphi}{dt} &= Z(t) 
\end{align*}

Substituting these values into \ref{eqn:x_sde} yields the SDE for the angle X
\begin{equation}
  \label{eqn:x_sde1}
 \frac{1}{2}\frac{d X}{dt} = -\mu_{0} \cos X + \omega - Y(t) \cos X + Z(t)
\end{equation}

As outlined in \cite{main}, the random parts of $Y(t)$ and $Z(t)$ are modelled as Ornstein-Uhlenbeck processes, with mean zero , time decay $\tau_{Y}$ and $\tau_{Z}$, and strengths $D{Y}$ and $D{Z}$
This yields the system of stochastic differential equations to model the angle X.
An Ornstein-Uhlenbeck process is a stochastic process that is a Gaussian process, a Markov process, and is temporally homogeneous.

\begin{equation}
 \begin{aligned}
 \frac{1}{2}\frac{d X}{dt} &= -\mu_{0} \cos X + \omega - Y(t) \cos X + Z(t) \\ 
 \frac{d Y}{dt} &= -\frac{Y}{\tau_{Y}} + \sqrt{\frac{2\sigma^{2}_{Y}}{\tau_{Y}}}dW_{Y} \\ 
 \frac{d Z}{dt} &= -\frac{Z}{\tau_{Z}} + \sqrt{\frac{2\sigma^{2}_{Z}}{\tau_{Z}}}dW_{Z} + \sqrt{\frac{2k^{2}}{\tau_{Z}}}dW_{Y} 
\end{aligned}
\end{equation}

Where $dW_{Y}$ and $dW_{Z}$ are uncorrelated weiner processes with $\langle dW_{Y,Z} \rangle = 0$ and $\langle dW^{2}_{Y,Z} \rangle = dt$, and cross-corrlation -$k \neq 0$.

Since the solution of an SDE is a markov process, obtaining the probability distribution of the underlying equilbrium solution requires many simulations of some numerical method and averaging over these.
This approach is outlined in the results section, but this direct numerical solution of the SDE's model converges very slowly.

The SDE model can be solved for a single sample path using the below scheme.

\begin{equation}
\begin{aligned}
 \label{eqn:sde_model_numeric}
 y_{t} &= y_{t-1} - \frac{y_{t-1}}{\tau_{Y}}\Delta t + \frac{D_{Y}}{\tau_{y}}W_{Y_{t}} \\ 
 z_{t} &= z_{t-1} - \frac{z_{t-1}}{\tau_{Z}}\Delta t + \frac{\sqrt{D_{Z}(1-k^{2})}}{\tau_{Z}}W_{Z_{t}} \\ 
 x_{t} &= x_{t-1} + \frac{\Delta t}{\gamma}\left[\omega + \left(- \cos x_{t-1} + k\left(\frac{D_{Z}}{D_{Y}}\right)^{1/2} \right) y_{t-1} + z_{t-1} \right] 
\end{aligned}
\end{equation}

\subsection{Fokker-Planck}\label{sec:der-sub}
Theory tells us that every stochastic differential equation has a corresponding partial differential equation that has a solution which is the probability density function of the stochastic differential equation.

The Fokker-Planck equation describes the evolution of conditional probability density for given initial states for a Markov process, since the random part of the SDE model is based on an Ornstein-Uhlenbeck process we should be able to 
define a Fokker-Planck for the SDE model here

There are two ways of dealing with the random terms in SDE's is the terms, the It\^o and Stratonovich interpretations. 
For multiplicative non-constant random terms each interpretations can yield different results.

The Itˆo interpretation requires the use of the It\^o calculus.  Stratonovich’s interpretation is based on the limit of the random terms
as the correlation time limits to zero, and it allows the use of the ordinary rules of calculus

Following the derivation outlined in \cite{risken} and \cite{hottovy}, the Stratonovich method is used to define the n-dimensional the Fokker-Planck equation as 

\begin{align}
 \frac{\partial P(\mathbf{X},t)}{\partial t} = \sum^{N}_{i=1}-\frac{\partial}{\partial X_{i}}(b(\mathbf{X})P(\mathbf{X},t)) - \sum^{N}_{i=1} \sum^{N}_{j=1} \frac{\partial}{\partial X_{i}\partial X_{j}}(\sigma(\mathbf{X})P(\mathbf{X},t))
\end{align}
where $\mathbf{X} = (X_1,X_2,...,X_N)$, $D^{1}(\mathbf{X}) = b(\mathbf{X})$ and $D^{2}(\mathbf{X}) = \sigma(\mathbf{X})$ and coefficients defined as 

\begin{align}
 D^{n}(\mathbf{Z}) = \frac{1}{n!}\frac{1}{\Delta t}\int^{\infty}_{-\infty}(\mathbf{Y} - \mathbf{Z})^{n}P(\mathbf{Y},\delta t | \mathbf{Z}) d \mathbf{Y}
\end{align}

Using this result and equating the coefficients of the \ref{eqn:sde_model_numeric}, yields the required Fokker-Planck equation




\begin{align}
 \label{eqn:fp}
 \frac{\partial P}{\partial t} &= \mathcal{L}_{OU}P - \frac{\partial}{\partial x}(VP) 
\end{align}
where P is the probability density function of the triple $(X,Y,Z)$ and 
\begin{align*}
  V(x,y,z) &= 2(\omega - y \cos x + z) \\
 \mathcal{L}_{OU} &= \frac{1}{\tau_{Y}} \frac{\partial}{\partial y}(y \circ) + \frac{1}{\tau^{2}_{Y}} \frac{\partial^{2}}{\partial y^{2}} + \frac{1}{\tau_{Z}} \frac{\partial}{\partial z}(z \circ)+ \frac{\rho}{\tau^{2}_{Z}} \frac{\partial^{2}}{\partial^{2} z} + \frac{2c\rho^{1/2}}{\tau_{Y}\tau_{Z}} \frac{\partial^{2}}{\partial y \partial z}\\
\end{align*}
where $c = \sqrt{\frac{k^{2}\tau_{Z}}{D_{Z}}}$, $\rho=\frac{D_{Z}}{D_{Y}}$

By solving \ref{eqn:fp} for the equalibrium distribution of $P(x,y,z)$, its follows that the PDF of the X-angle can be computed by finding the marginal distribution of X
\begin{align}
  \label{eqn:x-marginal}
  P_{X}(x) = \int_{-\infty}^{\infty} \int_{-\infty}^{\infty} P(x,y,z) dY dZ 
\end{align}
Similarly the PDF of the lypunov exponent is computed via a coordinate transformation on the joint marginal of $P_{XY}$

\begin{align}
  \label{eqn:lambda-marginal}
  P_{\Lambda}(\lambda) &= \int_{-\pi}^{\pi} P_{XY}\left(x,\frac{\lambda}{-2 \sin x}\right)\frac{1}{2|\sin x|}dX 
\end{align}

\subsubsection{Fokker-Planck numerical scheme}\label{sec:der-sub-fp}
The Fokker-Planck is solved using the same pseudo-spectral approach as was done for the vorticity and tracer concentration. One other aspect that must be addressed while solving the Fokker-Planck PDF is the \textit{CFL} condition.  
The general CFL condition for the n-dimensional case is defined as

\begin{align*}
  C=\Delta t\sum _{{i=1}}^{n}{\frac  {u_{{x_{i}}}}{\Delta x_{i}}}\leq C_{\max }.
\end{align*}

In this setting 
\begin{align*}
  C = \Delta t \min \{\Delta x, \Delta y, \Delta z\} = 0.1 V_{max}
\end{align*}
where $V_{max} = \left[\omega + \left( L_{y}/2 \right) \left( 1 + k\delta^{1/2}\right) + \left( L_{z}/2 \right) \right] / \gamma$.

This is a necessary condition for convergence when numerically solving certain classes of PDE's which includes the Fokker-Planck equation.


\label{sec:methods}

\section{Using Uncertainty Quantification methods to fit Fokker-Planck parameters}\label{sec:der-sub}

In \cite{main}, the values of the parameters in \ref{eqn:fp} are estimated using the moments of the $\omega$ and $\theta$, thus the model of the stochastic orientation dynamics 
are shown as $D_{Y} = 0.05/\tau$, $D_{Z} = 0.9/\tau$, $k = 0$, $w = 0$ and $\tau = 0.1\sqrt{\langle || \omega||^{2}_{2}\rangle}$

Uncertainty Quantification (UQ) methods try to encapsulate all the error and uncertainty in a models parameters so that the output can be evaluated and interpreted constrained by the 
inherent uncertainty.

Due to time constraints applying UQ methods to \ref{eqn:fp} was not attempted. A few Uncertainty Quantification methods as outlined in \cite{uqm} are breifly summaried, that could be applied to 
 to quantify the parameters. 

\subsection{Monte Carlo}
In this method the SDE is solved directly by randomly sampling from some distribution that the unknown parameters are assumed to come from, this makes the SDE deterministic and a solution can be found. 
This proceess is repeated $N$ times and average statistics can be computed. But this has very poor convergence. The Euler-Maruyama described in the result section is an example of this, but the parameters $D_Y$, $D_Z$, $k$ and $w$
are not treated as unknown parameters drawn from some distribution.

\subsection{Stochastic Collocation Method}
This similar to the vanilla Monte Carlo method except that random space is represneted by fewer points , each with a corresponding weight, which are used to calculate the mean and avergae from $N$ runs of the method.
This has exponential convergence.

\subsection{Stochastic Galerkin Method}
In this setting the solution is represented by a \textit{Polynomial Chaos Expansion}. The type of polynomial choosen depends on the assumed distribution of the unknown parameter,e.g for Gaussian random variables , Hermite polynomials are
used. This method also has exponential convergence.

