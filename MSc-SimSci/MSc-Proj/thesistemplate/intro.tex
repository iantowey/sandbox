%\documentclass[a4paper,10pt]{book}
%\usepackage[utf8]{inputenc}

%\begin{document}

\chapter{Introduction} \label{ch:intro}

%\section{new}
  
  This paper investigates properties of turbulent fluid flow. Turbulent fluid flow is a complex and requires advanced mathematics to describe this motion.
  The motion of a fluid is usually though as of a group of unit fluid elements with similar physical properties, the mathematics is modelled on how the fluid motion effects a 
  single fluid element. In a turbulent flow there is mixing of fluid particles, which are stretched, distorted and mixed in knots by the flow. In this setting,
  the motion of the fluid element is highly irregular and statistical averages are used to describe the various properties of the fluid.
  Turbulent fluid flow has important application in industrial mixing applications, examples outlined cite:mixingwiki and cite:fluidmixing.
      
  This paper investigates using a stochastic model to describe the \textit{mixing/turbulent} motion of a passive tracer in a fluid flow. 
  Rapid mixing or Turbulent flow is conventionally visualised as a cascade of large eddies breaking into successfully smaller eddies and the transfer of energy from these larger unstable eddies to smaller eddies.
  A passive tracer (such as a dye) is any fluid property we can measure to track fluid flow that does not influence the properties of the flow.
  Turbulent / mixing action stretches and distorts these fluid elements. The rate of stretching/distortion is described by a quantity known as the Lyapunov exponent. The solution to the stochastic model provides the 
  probability density function of the Lyapunov exponent. 
      
  In cite:main, a model of the orientation dynamics is derived based on the alignment dynamics of the tracer gradient with the straining direction of the flow. 
  The orientation dynamics model emerges from the standard advection-diffusion equation (ref eqn) for viscous fluids by introducting a vector field which describes the 
  the gradient of the tracer. The eigenvalues and eigenvectors of the rate-of-strain tensor associated with the orientation dynamics model are the basis from which the stochastic model for the Lyapunov exponent PDF is found.
  The stochastic fluctuations are modeled as as Ornstein-Uhlenbeck (OU) processes, this is an appropriate assumption as the analysis in cite:main introduces a external random forcing in the simulation. 
  The Fokker-Planck partial differential equation associated with the stochastic model is presented, which describes the stationary solution of the stochastic model. 
  The parameters of the Fokker-Planck were choosen via the moment of the vorticity and tracer concentration from the DNS. 
  One of the aims of this project is to try and use uncertainty quanification methods to fite these parameters more accurately.

  The aim of this paper is to fully dervice the equations presented in cite:main, numerically solve the vorticity, advection diffusion and the Fokker-Planck equations. Reproducing the simulation results in cite:main and
  use uncertainty quatification methods to fit the FP parameters more actually accurately.
  Uncertainty quantification (UQ) methods are planned to be used to verify and validate the results of the stochastic model against the output of direct numerical simulation of the vorticity/advection-diffusion equations. 
  This allows us to understand the expected uncertainty in the output of the model and quantify the error from the experiment. A number of open-source UQ libraries are investigated to see if they can be used to solve 
  the Fokker-Planck equation and fit the parameters to lead to a more accurate fit between the stochastic model and the direct numerical simulation output.

  First we review and detail the underlying theory, derive the stochastic model and present the numerical schemes.
%\end{document}
